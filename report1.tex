\documentclass[a4]{jsarticle}
\title{計算機科学実験及演習4 DBレポート1}
\author{里見 琢聞\\1029-26-3834}
\date{\today}
%-----------------------------------------------------------%
\begin{document}
\maketitle
DVDレンタル店での利用を想定したDVDレンタルサービスアプリケーションの製作に当たって、そのデータベース設計をおこなう。
\section{アプリケーションの説明}
本アプリケーションによって、利用者はレンタルDVDの管理や、貸出・返却の手続き、貸出履歴の検索などを行うことができる。設定は以下のとおりである。(後の変更可能性大)
\begin{itemize}
\item 貸出商品はDVDのみである。
\item 管理者により、バーコードの貼ったDVDをシステムに登録することによって、レンタル商品となる。このとき、同作品もIDによって区別する。
\item レンタルをするには、本人のアカウントをシステムに登録をし、貸出手続きを販売員に行ってもらうことでできる。
\item レンタルを行うと、貸出の日付などの詳細、料金、返却日などが記載された貸出明細書が発行される。
\item 選択可能なレンタル期間と、新作・準新作・それ以外の作品及びレンタル料金の関係は次のとおりである。一度の手続きで借りられるのは5本までで、レンタル中のDVDがある間に別のDVDを新しく借りることはできない。
\item 返却手続きは、店内の機械を操作することによって、次項の例外を除き客一人で行うことができる。
\item 返却日が過ぎると、超過日数により次の延滞金が発生する。この場合、返却手続きを販売員に行ってもらう必要がある。
\item 延滞の激しい利用者に対して、管理者がアカウントの凍結を行うことで、その後そのアカウントを使ってレンタルをすることができなくなる。
\item アカウントを持っている人は、お店のウェブページから、自分の現在の貸出情報を閲覧したり、他の作品のレンタル情報を検索することができる。
%-----------------------------------------------------------%
\section{利用者の役割の列挙と説明}
管理者、販売者、購入者
%-----------------------------------------------------------%
\section{役割ごとの機能の列挙と説明}
\subsection{管理者}
\begin{description}
\item[商品情報の新規登録]\\
新しく店に仕入れたDVDを、DVDに関する情報を入力してシステムに新しく登録する。
\item[商品情報の検索]\\
システム上の商品を、その属性や貸出履歴などによって検索する。
\item[商品情報の変更]\\
システム上に登録されている商品の情報を書き換える。
\item[商品情報の破棄]\\
システム上に登録されている商品を削除する。
\item[アカウント情報の検索]\\
名前や住所など、アカウントの属性でアカウントを検索する。
\item[アカウントの凍結]\\
特定のアカウントのレンタルサービスの利用をできなくする。
\end{description}
%-------------------------------------------%
\subsection{販売者}
\begin{description}
\item[商品の貸出手続き]\\
レンタルに必要な情報を入力することで、商品を貸出中にし、貸出明細書を発行する。
\item[商品の返却手続き]\\
返却される商品を受け取って、商品を貸出可能にする。
\end{description}
%-------------------------------------------%
\subsection{購入者}
\begin{description}
\item[アカウントの作成]\\
\item[商品のレンタル情報の検索]\\
\item[自分のレンタル情報の閲覧]\\
\item[返却手続き]\\
\end{description}
%-----------------------------------------------------------%
\section{実体関連図とその説明}
\subsection{実体関連図}
%-------------------------------------------%
\subsection{実体集合の説明}
\subsubsection{アカウント}
\begin{description}
\item[アカウントID]\\
\item[名前]\\
\item[住所]\\
\item[電話番号]\\
\item[レンタル履歴]\\
%---------------------------%
\subsubsection{レンタル商品}
%-------------------------------------------%
\subsection{関連集合の説明}
\end{document}
